
\addcontentsline{toc}{section}{\textcolor{cyan}{\textbf{Introduction générale}}}
\begin{flushleft}
\section*{\textcolor{cyan}{Introduction générale}}
 Les maladies chroniques sont responsables de 65\% des décès et demeurent la cause principale de tous les décès prématurés, où les maladies du cœur occupent la deuxième place. Les personnes atteintes de maladies chroniques ont besoin d’un suivi régulier de leurs états de santé. Souvent, elles ne sont pas en mesure de recevoir les soins nécessaires pour plusieurs raisons. Avec l'avènement de la technologie des télécommunications et d’internet, les systèmes de soins de santé se sont constamment améliorés avec le temps et de nombreux pays ont réalisé des gains importants d'espérance de vie de leurs populations. La télésurveillance s'est avérée capable d'améliorer considérablement les résultats du traitement de nombreuses maladies chroniques. \newline
 
 Notre projet vise à mettre en place un système d'e-santé basé sur la télésurveillance des signes vitaux de personnes souffrants de maladies cardiaques. Des capteurs connectés permettent de recueillir les signaux (ECG, PPG) et de transmettre les données vers un serveur puis l'utilisation d'un smartphone pour la visualisation. Ces données sont également accessibles à partir d'une plateforme web et peuvent être consultées en même temps par un médecin ou un autre membre du personnel médical. Notre système permettra d'éviter certaines complications de leur état de santé par la prise en 
 charge immédiate, de contribuer à la rapidité de fournir un diagnostic
  ou un traitement et d'envoyer des alertes par SMS en cas d'arythmie ou de niveau anormal de SpO2.\newline
 
 Dans le cadre de mon stage au sein de l'association MolbiolExpert, j'ai pour objectif de réaliser un prototype de système de télésurveillance pour le suivi des patients atteints de maladies cardiaques. Pour cela, j'utiliserai le langage C avec l'IDE Arduino pour mettre en œuvre un système de mesure de la saturation d'oxygène en temps réel. Ce système sera basé sur un microcontrôleur ESP8266 relié à un capteur MAX30102 (Oxymètre de pouls MAX30102) et un afficheur OLED. \newline
 
 

\end{flushleft}
\bibliographystyle{plain}
\newpage
	