
\addcontentsline{toc}{section}{\textcolor{blue}{ANNEXES}}

\section*{\textcolor{blue}{ANNEXES}}

\section*{\textcolor{cyan}{Programme en Arduino qui utilise l'oxymètre de pouls MAX30102 pour mesurer la température corporelle, la fréquence cardiaque et la saturation en oxygène, puis affiche ces mesures sur un écran OLED SSD1306 }}

\begin{flushleft}
	\begin{lstlisting}[style=CStyle]
		#include <Wire.h>
		#include <Adafruit_SSD1306.h>
		#include "MAX30105.h"
		#include <MAX30105.h>
		
		#define SCREEN_WIDTH 128
		#define SCREEN_HEIGHT 32
		Adafruit_SSD1306 display(SCREEN_WIDTH, SCREEN_HEIGHT, &Wire, -1);
		
		MAX30105 particleSensor;
		
		void setup() {
			display.begin(SSD1306_SWITCHCAPVCC, 0x3C);
			display.display();
			delay(2000);
			display.clearDisplay();
			
			Wire.begin(4, 5);
			particleSensor.begin(Wire, I2C_SPEED_FAST);
			particleSensor.setup();
			particleSensor.setPulseAmplitudeRed(0x0A);
			particleSensor.setPulseAmplitudeGreen(0);
		}
		
		void loop() {
			int32_t ir, red;
			float temperature, bpm, spo2;
			
			temperature = particleSensor.readTemperature();
			ir = particleSensor.getIR();
			red = particleSensor.getRed();
			bpm = particleSensor.getHeartRate();
			spo2 = particleSensor.getSpO2();
			
			display.clearDisplay();
			display.setTextSize(1);
			display.setTextColor(WHITE);
			display.setCursor(0, 0);
			display.println("Temperature: " + String(temperature) + " C");
			display.setCursor(0, 10);
			display.println("Heart Rate: " + String(bpm) + " bpm");
			display.setCursor(0, 20);
			display.println("SpO2: " + String(spo2) + " \%");
			display.display();
			
			delay(1000);
		}
		
		
	\end{lstlisting}
\section*{\textcolor{cyan}{Programme Arduino qui permet de connecter un module ESP8266 à AWS IoT Core}}
	
	\begin{lstlisting}[style=CStyle]
		#include <ESP8266WiFi.h>
		#include <WiFiClientSecure.h>
		#include <PubSubClient.h>
		#include <ArduinoJson.h>
		#include <time.h>
		#include "secrets.h"
		
		
		
		float temperature, bpm, spo2;
		
		unsigned long lastMillis = 0;
		unsigned long previousMillis = 0;
		const long interval = 5000;
		
		#define AWS_IOT_PUBLISH_TOPIC   "esp8266/pub"
		#define AWS_IOT_SUBSCRIBE_TOPIC "esp8266/sub"
		
		WiFiClientSecure net;
		
		BearSSL::X509List cert(cacert);
		BearSSL::X509List client_crt(client_cert);
		BearSSL::PrivateKey key(privkey);
		
		PubSubClient client(net);
		
		time_t now;
		time_t nowish = 1510592825;
		
		
		void NTPConnect(void){
			Serial.print("Setting time using SNTP");
			configTime(TIME_ZONE * 3600, 0 * 3600, "pool.ntp.org", "time.nist.gov");
			now = time(nullptr);
			while (now < nowish)
			{
				delay(500);
				Serial.print(".");
				now = time(nullptr);
			}
			Serial.println("done!");
			struct tm timeinfo;
			gmtime_r(&now, &timeinfo);
			Serial.print("Current time: ");
			Serial.print(asctime(&timeinfo));
		}
		
		
		void messageReceived(char *topic, byte *payload, unsigned int length){
			Serial.print("Received [");
			Serial.print(topic);
			Serial.print("]: ");
			for (int i = 0; i < length; i++)
			{
				Serial.print((char)payload[i]);
			}
			Serial.println();
		}
		
		
		void connectAWS(){
			delay(3000);
			WiFi.mode(WIFI_STA);
			WiFi.begin(WIFI_SSID, WIFI_PASSWORD);
			
			Serial.println(String("Attempting to connect to SSID: ") + String(WIFI_SSID));
			
			while (WiFi.status() != WL_CONNECTED)
			{
				Serial.print(".");
				delay(1000);
			}
			
			NTPConnect();
			
			net.setTrustAnchors(&cert);
			net.setClientRSACert(&client_crt, &key);
			
			client.setServer(MQTT_HOST, 8883);
			client.setCallback(messageReceived);
			
			
			Serial.println("Connecting to AWS IOT");
			
			while (!client.connect(THINGNAME)){
				Serial.print(".");
				delay(1000);
			}
			
			if (!client.connected()) {
				Serial.println("AWS IoT Timeout!");
				return;
			}
			// Subscribe to a topic
			client.subscribe(AWS_IOT_SUBSCRIBE_TOPIC);
			
			Serial.println("AWS IoT Connected!");
		}
		
		
		void publishMessage(){
			StaticJsonDocument<200> doc;
			doc["time"] = millis();
			doc["temperature"] = temperature;
			doc["Heart-Rate"] = bpm;
			doc["SpO2"] = spo2;
			char jsonBuffer[512];
			serializeJson(doc, jsonBuffer); // print to client
			
			client.publish(AWS_IOT_PUBLISH_TOPIC, jsonBuffer);
		}
		
		
		void setup(){
			Serial.begin(115200);
			connectAWS();
		}
		
		
		void loop(){
			
			temperature = 12.2;
			bpm = 15.02;
			spo2 = 80 ;
			
			if (isnan(temperature) || isnan(bpm) ||isnan(bpm) )  // Check if any reads failed and exit early (to try again).
			{
				Serial.println(F("Failed to read from DHT sensor!"));
				return;
			}
			
			Serial.println("Temperature: " + String(temperature) + " C");
			Serial.println("Heart Rate: " + String(bpm) + " bpm");
			Serial.println("SpO2: " + String(spo2) + " %");
			
			delay(2000);
			
			now = time(nullptr);
			
			if (!client.connected()){
				connectAWS();
			}
			else{
				client.loop();
				if (millis() - lastMillis > 5000){
					lastMillis = millis();
					publishMessage();
				}
			}
		}
		
	\end{lstlisting}
\end{flushleft}