\begin{flushleft}
	\section{\textcolor{cyan}{Mise en contexte, problématique et objectifs : }}
	
	\subsection{\textcolor{green}{Problématique :}}
	La mise en place d'un système de télésurveillance pour le suivi des patients atteints de maladies cardiaques répond à plusieurs problématiques. Tout d'abord, la disponibilité du personnel médical dans les centres de santé est souvent insuffisante, une situation qui s'est aggravée avec la pandémie de la Covid-19. Par ailleurs, les patients vivant dans des zones reculées rencontrent des difficultés pour accéder aux soins de santé, d'où l'importance d'un système de suivi à distance. En outre, la surveillance en temps réel des constantes vitales des patients est souvent négligée ou absente, alors que cela constitue un besoin essentiel. Enfin, la demande croissante de dispositifs médicaux portables et fiables pour une utilisation à domicile renforce la pertinence d'un système de télésurveillance convivial et continu pour surveiller les paramètres cliniques vitaux des patients atteints de maladies cardiaques.
	
	\subsection{\textcolor{green}{Objectifs :}}
	
	Ce projet vise à améliorer la qualité des soins offerts aux personnes souffrant de maladies cardiaques en mettant en place un système d'E-santé basé sur la télésurveillance des signes vitaux. Ce système permettra un suivi en temps-réel, ce qui contribuera à la rapidité de fournir un diagnostic précis en cas de situation anormale d'un de ces signes vitaux. L'objectif principal de ce système est donc de prévenir les complications en détectant rapidement les signes vitaux anormaux et d'offrir une intervention médicale appropriée. Cela aidera également à réduire les coûts des soins de santé en évitant les consultations inutiles et en offrant un suivi continu et personnalisé. En somme, la mise en place de ce système de télésurveillance des signes vitaux contribuera à une meilleure prise en charge des patients atteints de maladies cardiaques, en permettant un suivi plus précis, une intervention rapide en cas de besoin et en prévenant les complications graves.
\end{flushleft}